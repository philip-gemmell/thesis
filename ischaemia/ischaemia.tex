\documentclass[../thesis-main.tex]{subfiles}

\begin{document}

\chapter{Isch\ae{}mic Variation}
\label{ch:ischaemia}

Test: Normal: \ae{}

Capitals: \AE{}

Italics: \emph{\ae{}}

Small caps: \textsc{\AE{}, \ae{}}

\begin{aquote}{Source}
  {\fontencoding{T1}\fontfamily{pzc}\selectfont
   Quotation
  }
\end{aquote}
\rule{\linewidth}{0.25mm}

 \begin{quote}
  \emph{Section Outline}
 \end{quote}
 
 \section{Variation Within Isch\ae{}mia}
 \label{sec:ischaemia-rationale}
 Introduction. Outline the reasons for using model populations to investigate isch\ae{}mia (benign variation leading to malign variation). Mention limitations of current study re: cell \emph{versus} tissue.
 
 \section{Model adaptation to Isch\ae{}mia}
 \label{sec:ischaemia-methods}
 Methods. Outline adaptation of model populations to simulate isch\ae{}mic conditions. Retraining of \istim{}. Inhibiton of \ina{} and \ica{}. Present data for training value of \gkatp{}---different effects of different values, comparison with past literature. Outline stimulation protocol for Shannon and Mahajan populations, with reasons for different protocols.
 
 \subsection{Isch\ae{}mic Biomarkers}
 \label{subsec:isch-biomarkers}
 Outline different biomarkers being measured, and the utility of each one. Describe calculation of ERP, with justification. Mention link between dV/dt and conduction velocity, with references. Explore utility of APD90, ERP and PRR.
 
 \section{Population Response to Isch\ae{}mia}
 \label{sec:isch-population}
 Results---take from isch\ae{}mia paper.
 
 \subsection{APD\sub{90}}
 \label{subsec:isch-apd90-response}
 
 \subsection{ERP and PRR}
 \label{subsec:isch-erpprr-response}
 
 \subsection{Other Biomarkers}
 \label{subsec:isch-other-response}
 
 \section{Model Failure During Isch\ae{}mia}
 \label{sec:isch-modelFailure}

\end{document}